




\section{Traduction du Schéma Entité-Association}
	En traduisant le schéma entité-association on obtient les tables suivantes:
	\begin{itemize}
		\item Covoitureur(\underline{email}, nom, prenom, addresse, date\_de\_naissance, mdp, num\_permis, argent, parrain)
		\item Localisation(\underline{id\_localisation}, nom, adresse, description, ville)
		\item Etape(\underline{id\_etape}, heure\_prevue, heure\_reel, statut, date, trajet, est\_depart\_de, est\_arriver\_de, localisation)
		\item Trajet(\underline{id\_trajet}, date, cout, statut, vehicule, conducteur, nb\_km, duree)
		\item Vehicules(\underline{immatriculation}, modele, nb\_place, couleur, carburant, crit\_air, proprietaire)
		\item Sponsor(\underline{id\_sponsor}, nom, nature, duree\_requise, remmuneration)
		
		
		\item Sponsorise(\underline{id\_sponsor, immatriculation})
		\item Commente(\underline{email, id\_localisation}, commentaire)
		\item Evalue(\underline{source, cible}, note)
	\end{itemize}
	On a les clefs étrangères:
	\begin{itemize}
		\item Sponsorise(id\_sponsor) fais reférence à Sponsor(id\_sponsor)
		\item Sponsorise(immatriculation) fais reférence à Vehicules(immatriculation)
		\item Commente(email) fais reférence à Covoitureur(email)
		\item Commente(id\_localisation) fais reférence à Localisation(id\_localisation)
		\item Evalue(source) fais reférence à Covoitureur(email)
		\item Evalue(cible) fais reférence à Covoitureur(email)
		\item Covoitureur(parrain) fais reférence à Covoitureur(email)
		\item Vehicule(proprietaire) fais référence à Covoitureur(email)
		\item Trajet(vehicule) fais référence à Vehicules(immatriculation)
		\item Trajet(conducteur) fais référence à Covoitureur(email)
		\item Etape(trajet) fais référence  à Trajet(id\_trajet)
		\item Etape(est\_depart\_de) fais référence à Covoitureur(email)
		\item Etape(est\_arriver\_de) fais référence à Covoitureur(email)
		\item Etape(localisation) fais référence à Localisation(id\_localisation)
	\end{itemize}
	
	\section{Implémentation}	
	\subsection{Schéma de la base et Justification des choix}
	\begin{center}
		\begin{tabular}{|c|c|c|}
			\hline
			\multicolumn{3}{|c|}{Covoitureur}\\
			\hline
			Nom & Type & Contrainte \\
			\hline
			email & varchar(100) & PRIMARY KEY \\\hline
			nom & varchar(50) & NOT NULL \\\hline
			prenom & varchar(50) & NOT NULL \\\hline
			addresse & varchar(100) & NOT NULL \\\hline
			date\_de\_naissance & date & NOT NULL \\\hline
			mdp & char(64) & NOT NULL \\\hline
			num\_permis & char(12) & \\\hline
			argent & numeric(11, 2) & default 0 \\\hline
			parrain & varchar(100)	& REFERENCES Covoitureur(email) ON DELETE SET NULL \\\hline
			\multicolumn{3}{|c|}{CHECK (email  $< >$ parrain)}\\\hline
		\end{tabular}
		
		\begin{tabular}{|c|c|c|}
			\hline
			\multicolumn{3}{|c|}{Localisation}\\
			\hline
			Nom & Type & Contrainte \\
			\hline
			id\_localisation & serial & PRIMARY KEY \\\hline
			nom & varchar(50) & NOT NULL \\\hline
			addresse & varchar(100) & NOT NULL \\\hline
			description & varchar(500) & \\\hline
			ville & varchar(64) & NOT NULL \\\hline
		\end{tabular}
		\begin{tabular}{|c|c|c|}
			\hline
			\multicolumn{3}{|c|}{Etape}\\
			\hline
			Nom & Type & Contrainte \\
			\hline
			id\_etape & serial & PRIMARY KEY \\\hline
			heure\_prevue & time & NOT NULL \\\hline
			heure\_reel & time &  \\\hline
			status & bool & default false \\\hline
			date & date & NOT NULL \\\hline
			trajet & int & NOT NULL REFERENCES trajet(id\_trajet) ON DELETE CASCADE \\\hline
			est\_depart\_de & varchar(100) & REFERENCES Covoitureur(email) ON DELETE CASCADE \\\hline
			est\_arriver\_de & varchar(100) & REFERENCES Covoitureur(email) ON DELETE CASCADE \\\hline
			localisation & int & NOT NULL REFERENCES Localisation(id\_localisation) ON DELETE CASCADE\\\hline
			\multicolumn{3}{|c|}{CHECK((est\_depart\_de IS NULL AND est\_arriver\_de IS NOT NULL)} \\
			\multicolumn{3}{|c|}{OR (est\_depart\_de IS NOT NULL AND est\_arriver\_de IS NULL))}\\\hline
		\end{tabular}
		
		
		
		\begin{tabular}{|c|c|c|}
			\hline
			\multicolumn{3}{|c|}{Vehicules}\\
			\hline
			Nom & Type & Contrainte \\
			\hline
			immatriculation & char(7) & PRIMARY KEY \\\hline
			modele & varchar(55) & NOT NULL \\\hline
			nb\_places & int & NOT NULL \\\hline
			couleur & varchar(60) & NOT NULL \\\hline
			carburant & varchar(50) & NOT NULL \\\hline
			crit\_air & char(1) & NOT NULL CHECK(crit\_air IN ('E', '1', '2', '3', '4', '5')) \\\hline
			proprietaire & varchar(100) & NOT NULL REFERENCES Covoirureur(email) \\
			&&  ON DELETE CASCADE ON UPDATE CASCADE \\\hline
		\end{tabular}
		
		\begin{tabular}{|c|c|c|}
				\hline
			\multicolumn{3}{|c|}{Trajet}\\
			\hline
			Nom & Type & Contrainte \\
			\hline
			id\_trajet & serial & PRIMARY KEY \\\hline
			date & date & NOT NULL \\\hline
			cout & numeric(9, 2) & NOT NULL \\\hline
			statut & bool & default false \\\hline
			vehicule & char(7) & NOT NULL REFERENCES Vehicules(immatriculation) ON DELETE CASCADE \\\hline
			localisation & int & NOT NULL REFERENCES Localisation(id\_localisation) ON DELETE CASCADE \\\hline
			nb\_km & int & CHECK(nb\_km $>$= 0)  \\\hline
			duree & interval &\\\hline
		\end{tabular}
		
		\begin{tabular}{|c|c|c|}
			\hline
			\multicolumn{3}{|c|}{Sponsor}\\
			\hline
			Nom & Type & Contrainte \\
			\hline
			id\_sponsor & serial & PRIMARY KEY \\\hline
			nom & varchar(75) & NOT NULL \\\hline
			nature & varchar(500) & NOT NULL \\\hline
			duree\_requise & interval & NOT NULL \\\hline
			remuneration & numeric(11, 2) & NOT NULL\\\hline
		\end{tabular}
		
		\begin{tabular}{|c|c|c|}
			\hline
			\multicolumn{3}{|c|}{Sponsorise}\\
			\hline
			Nom & Type & Contrainte \\
			\hline
			id\_sponsor & int & REFERENCES Sponsor(id\_sponsor)\\\hline
			immatriculation & char(7) & REFERENCES Vehicules(immatriculation) \\\hline
			\multicolumn{3}{|c|}{PRIMARY KEY (id\_sponsor, immatriculation)}\\\hline
		\end{tabular}
		
		\begin{tabular}{|c|c|c|}
			\hline
			\multicolumn{3}{|c|}{Commente}\\
			\hline
			Nom & Type & Contrainte \\
			\hline
			email & varchar(100) & REFERENCES Covoitureur(email) \\
			& & ON DELETE CASCADE ON UPDATE CASCADE \\\hline
			id\_localisation & int & REFERENCES Localisation(id\_localisation) \\\hline
			commentaire & varchar(500) & NOT NULL \\\hline
			\multicolumn{3}{|c|}{PRIMARY KEY (email, id\_localisation)}\\\hline
		\end{tabular}
		
		\begin{tabular}{|c|c|c|}
			\hline
			\multicolumn{3}{|c|}{Evalue}\\
			\hline
			Nom & Type & Contrainte \\
			\hline
			source & varchar(100) & REFERENCES Covoitureur(email)  \\
			& & ON DELETE CASCADE ON UPDATE CASCADE\\\hline
			cible & varchar(100) & REFERENCES Covoitureur(email)  \\
			& & ON DELETE CASCADE ON UPDATE CASCADE\\\hline
			note & int & NOT NULL \\\hline
			\multicolumn{3}{|c|}{PRIMARY KEY (source, cible)}\\\hline
			\multicolumn{3}{|c|}{CHECK (source  $< >$ cible)}\\\hline
		\end{tabular}
	\end{center}
	Nous avons fais certain choix sur les types:
	\begin{itemize}
		\item Les mot de passent sont stocker comme des chaines de caractères d'exactement 64 caractères car ils seront hacher avant que leur forme hexadécimal soit stocker.
		\item  Nous séparons la date et les heures pour ne pas permettre de renseigné le décalage de deux jours
		\item Nous avons rajouter un attribut longueur pour pouvoir réaliser les demandes supplémentaires
	\end{itemize}

	\subsection{Contraintes non Garantie}
	Il y a des contraintes que nous n'avons pas pus implémentés dans le schéma:
	\begin{itemize}
		\item Il nous est impossible de vérifier que l'e-mail de l'utilisateur est valide
		\item Il en est de même pour les plaques d'immatriculations 
		\item  Nous ne pouvons pas nous assurer que les covoitureurs  déposent exactement deux étape pour un trajet
	\end{itemize}
	\pagebreak
	\section{Demande supplémentaire}
	Nous avons créer des vues pour répondre aux contraintes supplémentaire qui nous ont été demander
	\begin{itemize}
		\item Obtenir les 10 premiers points de rendez-vous les plus f
		\begin{lstlisting}[language=SQL]
CREATE VIEW TopLocation AS (
	SELECT adresse, ville, COUNT(id_etape) FROM etape
	NATURAL JOIN Localisation
	GROUPE BY adresse, ville
	ORDER BY COUNT(id_etape) DESC
	LIMIT 10
);
		\end{lstlisting}
		\item Le nombre de kilomètre parcourus pour chaque sponsor
		\begin{lstlisting}[language=SQL]
CREATE VIEW SponsorKm AS (
	SELECT id_sponsor, nom, SUM(nb_km) FROM Sponsor
	NATURAL JOIN Sponsorise NATURAL JOIN Vehicules
	JOIN Trajet ON vehicules = immatriculation
	GROUP BY id_sponsor, nom
);
		\end{lstlisting}
	\end{itemize}
	

 